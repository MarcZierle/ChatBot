\chapter{Installation}

Diese Installationsanweisungen wurden auf Ubuntu-Distributionen getestet\\
Superuser-Rechte sind hier vorausgesetzt\\
\\
\\
\\
Öffnen Sie die Konsole und geben Sie folgenden Befehl ein\\

\begin{framed}
\texttt{\$ sudo apt-get install python3 pip3 curl libpcre3 libpcre3-dev git}
\end{framed}
\hfill\\\\
\begin{tabular}{cl}
	python3 & Programm ist in Python geschrieben\\
	pip3 & Für die Installation der nötigen Module\\
	curl & Zum Installieren von stack \\
	libpcre3 & Zum Installieren von stack \\
	libpcre3-dev & Zum Installieren von stack \\
	git & Zum Runterladen von dem ChatBot und Duckling \\
\end{tabular}
\newpage
\begin{framed}
\texttt{\$ pip3 install rasa tensorflow spacy python-dotenv pillow \textbackslash \\ $\phantom{.}\qquad$ iso8601  icalendar}
\end{framed}
\hfill\\\\
\begin{tabular}{cl}
	rasa & Für Natural-Language-Processing und Rasa Core\\
	tensorflow & Benötigt für Rasa\\
	spacy & Benötigt für Rasa\\
	python-dotenv & Zum Laden von .env Dateien für die API-Keys \\
	pillow & Image-Library zum Erstellen von Bilddateien \\
	icalendar & Zum In-/Exportieren von .ical oder .ics Kalender-Dateien \\
	iso8601 & Zum Parsen von String zu Datetime-Objekten 
\end{tabular}
\\\\\\\\
Benötigt für Spacy
\begin{framed}
	\texttt{\$ python3 -m spacy download en\_core\_web\_md \\
		\$ python3 -m spacy link en\_core\_web\_md en}
\end{framed}
\hfill\\\\
Zum Runterladen von dem ChatBot und Duckling. Duckling wird benutzt um Zeiten aus Text zu extrahieren.
\begin{framed}
	\texttt{\$ git clone https://github.com/MarcZierle/ChatBot \\
		\$ cd ChatBot \\ 
		\$ git clone  https://github.com/facebook/duckling}
\end{framed}
\hfill\\\\
\newpage
Zum Installieren von Duckling. Dafür wird Stack benötigt was auch erst installiert werden muss. \texttt{\$ stack build} kann eine Weile zum Installieren benötigen.\\
\begin{framed}
	\texttt{\$ cd duckling \\
	\$ curl -sSL https://get.haskellstack.org/ | sh \\
	\$ stack init \\
	\$ sudo stack build \\
	\$ cd ..}
\end{framed}
\hfill\\\\


Als nächstes muss d Rasa Modell trainiert werden. Dazu einfach das Shell-Skript \texttt{train\_rasa\_model.sh} ausführen. \\
\begin{framed}
	\texttt{\$ ./trains\_rasa\_model.sh}
\end{framed}
\hfill\\
Der ChatBot Ordner muss noch dem \texttt{PYTHONPATH} hinzugefügt werden mit z.B \\
\begin{framed}
	\texttt{\$ export PYTHONPATH="/home/ChatBot"}
\end{framed}
\hfill\\

Der absolute Pfad des ChatBot Ordners muss auch noch in der \texttt{settings.py} eingetragen werden. \\
\begin{framed}
	6$\qquad$ ...\\
	7$\qquad$\texttt{env\_path = Path('/home/ChatBot') / '.env'}\\
	8$\qquad$ ...
\end{framed}
\hfill\\

Danach muss der Help Server für die Custom Actions für Rasa und für Duckling mit \texttt{run\_help\_servers.sh \&} gestartet werden.\\
 \begin{framed}
 	\texttt{\$ ./run\_help\_servers.sh \&}
 \end{framed}
 \hfill\\
 
Zum Starten des Programms dann die \texttt{main.py} starten.\\
  \begin{framed}
 	\texttt{\$ python3 main.py}
 \end{framed}
 \hfill\\\\
 
Nach dem Beenden der \texttt{main.py} kann der Help Server Prozess mit \texttt{\$ pkill rasa} beendet werden.
