\chapter{Fazit}

Für beide von uns es das erste Mal, einen Chatbot, im Rahmen des Semesterprojektes "Dialoge mit Computern", zu programmieren. Wir haben dadurch beide ein breites Verständnis für den Aufbau von Conversational Agents und deren zugrundeliegende Vorverarbeitung durch NLP erhalten.\\

Im Nachhinein hat sich die Verwendung von Telegram in Kombination mit RASA als durchaus richtige Entscheidung erwiesen. Telegram bietet mit seiner API und Handy-App bereits eine solide Grundlage, für die Interaktion zwischen Chatbot und Nutzer. RASA im Back-End hat uns durch seine recht hohe Komplexität anfangs viel Schwierigkeiten gemacht, durch die zahlreichen Quellen an Information, Dokumentation, sowie Foren andere RASA-Benutzer, konnten wir uns einen guten Überblick über viele Funktionen verschaffen.\\

Retrospektiv lässt sich sagen, dass wir zu viel Zeit mit der Einbindung von Telegram in Python verbracht und uns zu wenig auf das Verbessern der Inhaltserkennung von RASA konzentriert haben. Dies zeigt sich in unserer eigenen, aber auch in der Wahrnehmung der Nutzer. Das Problem, dass Adressen mit Uhrzeiten u.u. verwechselt werden, ist eines, mit welchem leider wir bis heute zu kämpfen haben. Die Evaluation hat ebenfalls gezeigt, dass Nutzer verständlicher Weise eine andere Sicht auf die Antworten unseres Bots haben, und es dadurch zu Missverständnissen und Benutzungsprobleme geben kann. Ein früheres Testen unter realen Bedingungen, hätte dem möglicherweise zuvor kommen können.\\

Alles in Allem sind wir zufrieden mit unserem Ergebnis, einen Terminplanenden und -optimierenden Chatbot vorzeigen zu können, und bedanken uns für diese tolle Chance. 
